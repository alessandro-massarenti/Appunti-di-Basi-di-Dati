\documentclass[a4paper, 11pt]{report}

\usepackage[utf8]{inputenc}
\usepackage{mathtools}
\usepackage{amsmath}
\usepackage{amsfonts}
\usepackage{graphicx}
\usepackage{listings}
\usepackage{xcolor}

\usepackage[italian]{babel}

\usepackage{hyperref}
\usepackage{geometry}
 \geometry{
 a4paper,
 left=25mm,
 right=25mm,
 top=20mm,
 bottom=20mm,
 }
 

\definecolor{codegreen}{rgb}{0,0.6,0}
\definecolor{codegray}{rgb}{0.5,0.5,0.5}
\definecolor{codepurple}{rgb}{0.58,0,0.82}
\definecolor{backcolour}{rgb}{0.95,0.95,0.92}

 
\lstdefinestyle{pretty_sql}{
    backgroundcolor=\color{backcolour},   
    commentstyle=\color{codegreen},
    keywordstyle=\color{magenta},
    numberstyle=\tiny\color{codegray},
    stringstyle=\color{codepurple},
    basicstyle=\ttfamily\footnotesize,
    breakatwhitespace=false,         
    breaklines=true,                 
    captionpos=b,                    
    keepspaces=true,                 
    numbers=left,                    
    numbersep=5pt,                  
    showspaces=false,                
    showstringspaces=false,
    showtabs=false,                  
    tabsize=2,
    language=sql
}

\lstset{style=pretty_sql}

\newtheorem{assioma}{Assioma}

\newtheorem{exmp}{Example}[section]

\setlength{\parskip}{1em}
\setlength{\parindent}{0pt}


\title{Appunti di \textit{"Basi di Dati"}}
\author{Alessandro Massarenti}
\date{September 2021}

\begin{document}

\maketitle

\tableofcontents

\chapter{Il modello relazionale}

Il modello di base di dati relazionale è il modello più utilizzato per rappresentare ed organizzare una base dati. Questo modello si basa sul concetto matematico di relazione, ovvero la connessione tra dati è relazionata per mezzo dei valori.

Quindi, su dominio:
\[D_1, D_2, ..., D_n\]

Posso avere relazioni:

\[D_1  \times D_3\]
\[D_2  \times D_3\]
\[D_4 \times D_1\]

\section{Caratteristiche}
Questo tipo di relazione e quindi di modello ha varie caratteristiche interessanti e peculiari.

\begin{itemize}
    \item Non c'è ordine tra le righe
    \item non c'è ordine tra le colonne
    \item le righe sono tutte diverse tra loro
    \item le intestazioni sono diverse tra loro
    \item i valori contenuti nelle colonne sono dello stesso dominio.
\end{itemize}

Inoltre, come dicevamo prima la connessione tra le varie relazioni è fatta per mezzo dei valori. Sarà quindi possibile collegare i dati e creare un'informazione più estesa ed importante.


\begin{exmp}\

    \includegraphics[width=\textwidth]{img/con_tra_rel.png}
\end{exmp}


\section{Tuple}

\begin{assioma}
Data la relazione $ R(A_1 , ..., A_n)$, una tupla è una funzione che definisce la relazione tra i vari dati. 
\end{assioma}

Le tuple vengono rappresentate come le righe delle tabelle in un database relazionale. Al singolare, una riga si chiama tupla.

Vedremo poi che una tupla è il risultato di una query, quindi non è solamente una riga della tabella se per ottenere più informazione uniamo i dati presenti in più tabelle.


%%Non chiaro
\begin{assioma}
Istanza di relazione su uno schermo $R(x)$ un insieme di tuple su $X$
\end{assioma}
%%------

\section{Valore NULL}

Il valore NULL è un valore che non fa parte del dominio, però è utile per definire dei campi in cui i dati non sono presenti.

\begin{description}
	\item[N.B.] Bisogna usare NULL con attenzione, perchè non facendo parte del dominio potrebbe dare comportamenti inattesi.
\end{description}

Per non tollerare valori nulli nel mio dominio dovrò utilizzare dei costraint.

\section{Risoluzione delle ridondanze}

Per risolvere le ridondanze, ovvero la duplicazione di dati nelle tabelle, si utilizzano i vincoli.

\section{Vincoli di integrità}

\subsection{Rigetto semantiche scorrette}
Nei database relazionali posso rigettare semantiche scorrette.

\includegraphics[width=\textwidth]{img/vincoli_di_integrità.png}

Alcuni tipi di vincoli sono supportati dal DBMS\footnote{Database management system}
Altri vincoli non sono supportati dal DBMS e vengono quindi mantenuti client side.

\subsection{Vincoli inter-relazionali}

I vincoli inter-relazionale è un vincolo che coinvolge più tabelle.

Se controllo l'integrità solo tramite vincoli di questo genere potrei però avere delle ridondanze, per questo si usa il vincolo di chiave.

\section{Vincoli di chiave}
Il vincolo di chiave dice che non posso avere tuple con i campi chiave uguali, ad esempio il codice fiscale che non può essere lo stesso per più persone.

Le chiavi verranno usate per gestire le relazioni tra tuple.

\subsection{Super-chiavi}

Un insieme $\mathbb{K}=\left\{K_1 , ..., K_n \right \} $ è una super chiave se non esistono tuple con gli stessi valori.

Una super-chiave identifica le tuple di una relazione.

Esiste infatti sempre almeno una super-chiave ed è \textbf{"tutta la tupla"}

\subsection{Chiavi}

Vedremo più in dettaglio le chiavi nel capitolo dedicato, capitolo \ref{chap:chiavi}

Una chiave è una super-chiave minimale quando:

\[\mathbb{K} \text{super-chiave è chiave} \leftrightarrow \forall k \mid k-k \text{ Non è superchiave} \]

Questa cosa va controllata che sia valida per ogni valore possibile e non solo per i valori attuali.

\begin{exmp}

Qui abbiamo un esempio delle chiavi.

\includegraphics[width=\textwidth]{img/esempio_chiavi.png}

In questo caso infatti $(NomeAmico,NomeLibro)$ è chiave se non posso prestare allo stesso amico lo stesso libro più volte.

Nella seconda parte dell'esempio invece se $(NomeLibro,Data)$ è chiave non posso prestare lo stesso libro più volte nello stesso giorno.
\end{exmp}



\chapter{Chiavi} \label{chap:chiavi}

\section{Chiavi primarie}
La chiave primaria è la chiave principale della tabella, serve per identificare l'unicità di una riga in una tabella.

Se la chiave è formata da più campi deve essere riportata tutta quando è utilizzata come chiave esterna.

La chiave primaria non ammette valori nulli.

\section{Chiavi esterne}
Una chiave esterna deve comparire come chiave primaria nella tabella referenziata.

Una chiave esterna può essere anche \textit{NULL}

\includegraphics[width=\textwidth]{img/chiave_esterna.png}

\subsection{Eliminazione di chiavi esterne}

L'eliminazione di chiavi esterne è una situazione molto pericolosa e piena di insidie, va quindi eseguita con cura per mantenere l'integrità del dominio che si sta rappresentando nella base dati.

Le modalità di eliminazione di dati inseriti nel contesto delle chiavi esterne sono di vari tipi:

\begin{itemize}
    \item On delete cascade, ovvero elimino tutti i dati che utilizzano la chiave esterna.
    \item Assegnazione di \textit{NULL}
    \item Errore di eliminazione, ovvero devo prima eliminare manualmente i dati che utilizzano la chiave esterna.
\end{itemize}

\section{Super-chiavi}

Le super-chiavi permettono di capire di quale riga stiamo parlando. Sarà importante che una riga sia sempre super-chiave, questo determina infatti la non ridondanza dei dati nella tabella.

\begin{description}
	\item[N.B.] Più la super-chiave è piccola, e interessa quindi meno campi, meglio è, infatti più la super-chiave è piccola più è ampio l'insieme di dati che possiamo salvare.
\end{description}
\chapter{Algebra relazionale}

L'algebra relazionale è un linguaggio atto a definire in maniera chiara e matematica le relazioni tra i dati.

Abbiamo quindi la differenza tra:

\begin{itemize}
    \item DDL: data definition language
    \item DML: data manipulation language
\end{itemize}

Le relazioni saranno definite come il semplice prodotto cartesiano tra insiemi.

\section{Operatore Unione}
L'operatore di unione è l'unione matematica di due insiemi, si avrà quindi come risultato tutti gli elementi degli insiemi uniti togliendo i duplicati.

Questo operatore è rappresentato dal simbolo $\cup$.

\section{Operatore intersezione}
L'operatore di intersezione sono, come in matematica, gli elementi comuni agli insiemi intersecati.

Questo operatore è rappresentato dal simbolo $\cap$.
\section{Differenza}
Considerando due insiemi, $A$ e $B$, l'operazione di differenza ritorna solo gli elementi presenti in $A$ e non in $B$.

Questo operatore è rappresentato dal simbolo $\subset$.

\section{ridefinizione}

Ho una ridefinizione quando cambio un'intestazione della tabella.

In inglese questa operazione viene definita come \textit{"Rename"}.

Questo operatore è rappresentato dal simbolo $\rho $\footnote{Rho}.

\section{selezione}

Quando prendo un sottoinsieme delle tuple faccio una selezione.

Rappresento questo operatore con $\sigma $\footnote{Sigma}

\begin{exmp}

Ecco qualche selezione d'esempio:

$\sigma \text{stipendio} > 50$;

$\sigma \text{stipendio} > 50 \wedge \text{nome}=\text{"Mario"}$;
\end{exmp}

\section{proiezione}

Sto facendo una proiezione quando prendo un sottoinsieme della tupla. Le proiezioni eliminano i duplicati.


\begin{description}
	\item[N.B.] Siccome eliminano i duplicati è bene farlo sulle super-chiavi.
\end{description}

Le proiezioni si rappresentano con il simbolo matematico $\Pi$\footnote{Pi} ovvero project.

%Da espandere di molto la sezione
\section{join}

Quando faccio una join prendo 2 relazioni e unisco le tuple su attributi uguali.

Un join è completo se faccio proiezioni sugli operandi riottengo i primi, non è completo se alcuni campi non corrispondono.

Il join ri rappresentano con il simbolo $\bowtie$.

Esistono poi più tipologie di join:
\begin{itemize}
    \item join esterno
    \item semi join
    \item join cartesiano, detto anche join naturale
    \item theta join
\end{itemize}



\subsection{Join esterno}

In un join esterno mantengo gli $R_1$ senza match mettendo i corrispondenti a \textit{NULL}

\begin{exmp}
\[ R_1 \bowtie _{LEFT} R_2  \]
\end{exmp}

\subsection{Semi join}

\begin{exmp}
\[ R_1 \bowtie _{SEMI} R_2 = \Pi _R1\left( R_1 \bowtie R_2 \right)   \]
\end{exmp}

\subsection{Join cartesiano, detto anche join naturale}

Join senza match tra attributi

\subsection{Theta join}
\[ R_1 \bowtie _{COND} R_2 = \sigma cond \left(R_1 \bowtie R_2 \right) \]

\chapter{Viste}\label{chap:viste}

Le viste sono una rappresentazione differente degli stessi dati, servono per rendere più leggibile l'informazione proveniente da alcuni dati e nei DBMS aiuta ad ottimizzare l'accesso in lettura alle informazioni, poichè in caso di join queste vengono precaricare.

\section{Tipi di viste}

Di viste ne esistono principalmente due tipi, queste viste hanno ognuna pregi e difetti.

\begin{itemize}
    \item viste materializzate
    \item viste virtuali
\end{itemize}

\subsection{Viste materializzate}
Sono viste memorizzate nel database e non hanno ricalcoli. Sono tendenzialmente delle viste ridondanti e non sempre sono supportate.

\subsection{Viste virtuali}
Sono alias per delle query e vengono ricalcolate.

\section{Ottimizzazione delle viste}

Chiaramente Esistono modi buoni e modi pessimi per utilizzare le viste, vediamo quindi come ottimizzare il loro utilizzo.

\subsection{Equivalenza di espressione}

$X(Y+Z) = X Y + X Z$

Ottimizzazione query

$ \sigma _{cond} \left( R_1 \bowtie  R_2 \right) = R_1 \bowtie \sigma _{cond} R_2$ se cond è indipendente


%Non ho capito cosa si intende in questo caso.


\subsection{Modifiche tramite viste}
È possibile fare modifiche tramite viste solo su join complete.
\chapter{SQL}

SQL sta per Structured Query Language. Questo linguaggio serve a definire dei comandi da inviare al DBMS, esso ci risponderà con informazioni relative alla query.

Tramite l'SQL si possono inserire, rimuovere e modificare dati, inoltre tramite questo linguaggio si possono creare le strutture e aggiungere legami tra tabelle e dati.

\section{Create table}

Crea una tabella, ovvero uno schema di relazione tra domini e dati.

\begin{lstlisting}
CREATE TABLE nome (
    nome_attr TIPO VINCOLI,
    FOREIGN KEY(attributo) REFERENCES Tabella(attributo),
    UNIQUE(attributo),
    )
\end{lstlisting}

\subsection{Domini elementari}
I domini elementari sono i domini base che può avere ogni dato inserito.

\begin{itemize}
    \item Stringhe
    \begin{itemize}
        \item char
        \item varchar
    \end{itemize}
    \item number
    \begin{itemize}
        \item int
        \item float
        \item ecc
    \end{itemize}
\end{itemize}

\subsection{Domini customizzati}

Posso anche creare domini customizzati, li creo con la seguente sintassi.

\begin{lstlisting}
CREATE DOMAIN nome
AS TIPO DEFAULT valore
CHECK(condizione)

\end{lstlisting}


\subsection{Vincoli}

Posso avere vari vincoli da aggiungere ai domini e ai dati presenti nella tabella.

\begin{lstlisting}
NOT NULL
UNIQUE
PRIMARI KEY
CHECK
REFERENCES
FOREIGN KEY
\end{lstlisting}


\subsection{Update e Delete}

Posso inoltre decidere cosa succede per i dati gerarchicamente inferiori in caso di aggiornamento o rimozione dati. Per scrivere questa decisione la sintassi è la seguente:

\begin{lstlisting}
ON <update | delete> <cascade | set null | set default | no action>
\end{lstlisting}

\section{Interrogazione del DBMS}

\begin{lstlisting}
    SELECT attributi
    FROM tabela
    WHERE condizione
\end{lstlisting}

Gli attributi sono rinominabili per poter rendere più leggibile il risultato dell'interrogazione. Per rinominarli si usa:

\begin{lstlisting}
    attributo as nomeattributo
\end{lstlisting}

Questi comandi SQL trovano le loro controparti in algebra relazionale come segue:

\begin{itemize}
    \item attributi sono $\Pi$
    \item condizione sono $\sigma$
    \item rinominazione è $\rho$
\end{itemize}

\begin{description}
	\item[N.B.] Una differenza molto importante tra SQL ed algebra relazionale è che nelle select non c'è collasso, se desidero questa funzionalità in più dovrò utilizzare il la keyword DISTINCT.
 \end{description}

\subsection{Pattern di ricerca}
Quando faccio un'interrogazione del database e utilizzo il modificatore WHERE, posso utilizzare dei pattern per la ricerca, uno dei più utilizzati è \textit{LIKE}, si utilizza come segue:

\begin{lstlisting}
    nome LIKE "pattern"
\end{lstlisting}

\begin{itemize}
    \item Il simbolo "-" significa, "qualsiasi carattere"
    \item Il simbolo "\%" significa, "qualsiasi sequenza di caratteri(anche vuota)
\end{itemize}

Qualasiasi confronto con \textit{NULL} è falso.

\subsection{join}

Una clausula di join si usa per combinare le righe di due o più tabelle, basandosi su una colonna di relazione tra loro.

In algebra relazionale questo concetto è definito da $\bowtie$.


\subsubsection{join esplicito}

Posso eseguire left join, right join e full join

\subsection{order by}

In un modello relazionale, i dati non hanno ordine. Quando si esegue una query si può quindi decidere di dare ad essi un ordine per renderli più leggibili agli umani. Per riordinare i dati la keyword che si utilizza è \textit{ORDER BY}.

Tramite \textit{ORDER BY} è possibile riordinare colonna per colonna, si potrà poi decidere se in ordine crescente o decrescente.

\subsection{operatori}

Alcuni operatori utilizzabili in SQL sono: 

\begin{itemize}
    \item \textit{COUNT}
    \item \textit{MIN}
    \item \textit{MAX}
    \item \textit{AVG}
    \item \textit{SUM}
\end{itemize}

Questi operatori si può abbastanza semplicemente capire cosa fanno.

\subsubsection{Operatori aggregati e \textit{NULL}}

Gli operatori aggregati ignorano i valori \textit{NULL}, anche count ignora i valori nulli.

\subsection{Raggruppamenti}
Tramite SQL è inoiltre possibile aggregare e raggruppare i risultati delle query, per farlo si usa \textit{GROUP BY}

\subsubsection{Union}


\subsubsection{Except}
\subsubsection{Intersect}
\subsubsection{In}
\subsubsection{Any}
\subsubsection{Exist}

\section{Concetti avanzati di SQL}

\subsection{Vincoli di integrità generici}

Tramite SQL posso creare dei vincoli di integrità generici, questi vincoli controlleranno che non ci siano condizioni non rispettate.

Per utilizzare questo strumento tramite SQL si utilizza:

\begin{lstlisting}[caption= Utilizzo di CHECK]
    CHECK (<condizione>)
    
    --oppure ad esempio
    
    CHECK(lordo = netto + ritenute)
\end{lstlisting}

\subsection{Creazione di viste}

In SQL si può svuiluppare il concetto di vista che si è osservato al capitolo \ref{chap:viste}.

Per creare una vista tramite SQL la dicitura è la seguente:

\begin{lstlisting}[caption=Creazione di viste]
    CREATE VIEW <nome_vista>[<lista_attributi>] AS SELECT
    
    WITH CHECK OPTION
    
    --Il check permette update,
    --ovvero la tupla modificata rimane nella lista.
    
    SELECT * FROM <nome_vista>
\end{lstlisting}

\subsection{Coalesce}

\subsection{null if}

\subsection{CASE}

\subsection{Controllo degli accessi}

\chapter{Progettazione di una base di dati}

Progettare una base di dati non è un compito semplice. Progettare una base di dati richiede più fasi finalizzate a raccogliere e raccontare la realtà in modo che sia semplicemente accessibile

Le fasi tecniche sono:
\begin{enumerate}
    \item Studio di fattibilità
    \item Raccolta e analisi dei requisiti
    \item progettazione delle funzionalità e dei dati manipolati
    \item Realizzazione
    \item Validazione e collaudio
    \item Funzionamento
\end{enumerate}

A grandi linee le due fasi che si hanno quando si progetta una base dati sono l'analisi e la progettazione vera e propria. Durante la fase di analisi si lavora relativamente alla progettazione concettuale. Durante la fase di progettazione vera e propria si lavorerà invece al modello logico e al modello fisico.

Nell'analisi si vede quindi "\textbf{che cosa si modella}", nella progettazione si vede il "\textbf{come si modella}"

\section{Lo schema ER}

Lo schema Entità Relazione è la visualizzazione più usata per rappresentare una base dati. Questo modello verrà utilizzato per rendere grafico lo schema concettuale e lo schema logico nei passaggi seguenti.


\subsection{I costrutti del ER model}

\paragraph{Entità} Se ha proprietà significative e descrive oggetti con esistenza autonoma.
\paragraph{Attributo} Se è semplice e non ha proprietà.
\paragraph{Relazione} Se correla due o più concetti.
\paragraph{Generalizzazione} Se è caso particolare di un altro.


In questo schema i rettangoli rappresenteranno le entità, le relazioni saranno rappresentate da dei rombi e queste saranno collegate alle entità tramite dei vertici\footnote{Le linee}.

In questo modello sarà importante scegliere \textbf{nomi espressivi} ce siano \textbf{singolari}. Per \textbf{le relazioni} il requisito è lo stesso ma sarà importante \textbf{sostantivarle}.

Queste scelte di nomi saranno utili per comprendere più facilmente cosa si andrà a fare senza forzatamente dare un orientamento alla direzione delle relazioni.

\subsection{Cardinalità delle relazioni e degli attributi}

%Da rivedere
Quando definisco le relazioni e gli attributi dovrò definirne anche la cadinalità, ovvero quante di quelle relazioni possono esserci.


Si definiscono con una coppia di valori tra parentesi, saranno il primo la quantità minima ed il secondo la quantità massima. ad esempio (2,6).

\includegraphics[width = \textwidth]{img/cardinalita_relazioni.png}

solitamente si usano numeri o identificatori come 0,1,n.

Posso definire cardinalità anche relativamente agli attributi.

\includegraphics[]{img/cardinalita_attributi.png}

\subsection{Le relazioni}

Esistono vari tipi di relazioni:
\begin{itemize}
    \item binarie
    \item n-arie
    \item ricorsive
\end{itemize}

\paragraph{Relazioni binarie} Una relazione è binaria quando si rapporta a due entità e ne definisce un rapporto.

\paragraph{Relazioni n-arie} Se ho una relazione tra tre\footnote{Ad esempio in questo caso è ternaria} o più entità la relazione sarà n-aria. Un esempio di relazione ternaria è se ad esempio do due volte lo stesso esame avrò tre entità: \textit{persona}, \textit{esame1}, \textit{esame2}

\paragraph{Relazioni ricorsive} Posso avere relazioni ricorsive in molti casi, ad esempio se ho un prodotto che può essere composto da più prodotti ho una relazione ricorsiva con l'entità \textit{prodotto}.

\subsubsection{Gli attributi}

Gli attributi descrivono un' entità. Nel diagramma ER si definiscono con due simboli:

\includegraphics{img/attributi.png}

Il simbolo vuoto indica un'attributo normale, un attributo pieno indica un attributo chiave.

\paragraph{Gli attributi chiave} Un attributo chiave è un attributo che identifica in maniera univoca un oggetto appartenente a quell'entità. Per un cittadino italiano la chiave è il codice fiscale e ogni cittadino è identificato da questa chiave.

\paragraph{Attributi composti} Oltre agli attributi singoli descrivibili semplicemente con il simbolo vuoto posso anche voler descrivere attributi composti, ad esempio un indirizzo è un attributo composto da: Via, numero, cap, città, ecc...


\subsection{Design patterns}
Questi sono alcuni dei design patterns più utilizzati quando si organizza uno schema ER. Questi pattern sono comodi perchè semplificano la creazione e la lettura di un ER graph.

\subsubsection{Reificazione di attributo di entità}
Reificare significa prendere l'astratto per concreto, ovvero considerare concetti, categorie, idee, rapporti astratti alla stregua di oggetti concreti.

In questo caso significa espandere un attributo trasformandolo in entità. Questo è comodo perché ci consente di dare all'attributo altri sotto attributi e in caso di bisogno lo si potrà anche mettere in relazione con altre entità.
\begin{center}
\includegraphics[width=0.5\textwidth]{img/reificazioneDiAttributoDiEntita.png}
\end{center}

\subsubsection{Part-of (Has-a)}
Quando un'entità non può esistere senza una sua sovra-entità, questa entità è "\textit{part-of}" della sovra-entità.

\begin{center}
    \includegraphics[width=0.7\textwidth]{img/partOf.png}
\end{center}

\subsubsection{Istance-of - Is-a}

Quando un'entità è una specifica versione di un'altra entità è un "\textit{Istance-of}" dell'altra entità.

\begin{center}
    \includegraphics[width=0.7\textwidth]{img/istanceOf.png}
\end{center}

\subsubsection{Reificazione di relazione binaria}

\begin{center}
    \includegraphics[width=0.7\textwidth]{img/reificazioneDiRelazioneBinaria.png}
\end{center}

\subsubsection{Reificazione di relazione ricorsiva}

\begin{center}
    \includegraphics[width=0.7\textwidth]{img/reificazioneDiRelazioneRicorsiva.png}
\end{center}


\subsubsection{Reificazione di attributo di relazione}

\begin{center}
    \includegraphics[width=0.7\textwidth]{img/reificazioneDiAttributoDiRelazione.png}
\end{center}

\subsubsection{Caso particolare}


\begin{center}
    \includegraphics[width=0.7\textwidth]{img/casoParticolare.png}
\end{center}
\subsubsection{Storicizzazione di un concetto}

\begin{center}
    \includegraphics[width=0.7\textwidth]{img/storicizzazioneDiConcetto.png}
    \includegraphics[width=0.7\textwidth]{img/storicizzazioneDiConcetto2.png}
\end{center}

\subsubsection{Evoluzione di un concetto}

\begin{center}
    \includegraphics[width=0.7\textwidth]{img/evoluzioneDiConcetto.png}
\end{center}

\subsubsection{Relazione ternaria}

\begin{center}
    \includegraphics[width=0.7\textwidth]{img/relazioneTernaria.png}
\end{center}

\subsubsection{Reificazione di relazione ternaria}
\begin{center}
    \includegraphics[width=0.7\textwidth]{img/reificazioneDiRelazioneTernaria.png}
    \includegraphics[width=0.7\textwidth]{img/reificazioneDiRelazioneTernaria2.png}
\end{center}

\section{Analisi}

Durante questa fase, lo scopo è quello di comprendere al meglio e nella maniera più aderente possibile cosa si dovrà modellare poi nelle fasi successive. I passaggi sono inquadrati nel seguente modo:

\begin{enumerate}
    \item Acquisizione dei requisiti
    \item Analisi dei requisiti
    \item Costruzione dello schema concettuale
    \item Costruzione del glossario
\end{enumerate}

In questa fase devo quindi acquisire i requisiti per la realizzazione, analizzare i dati raccolti e dovrò da queste informazioni ottenute generare uno schema relazionale ed un glossario dei termini.

\subsection{Acquisizione dei requisiti}

Questo è il momento in cui dovrò ascoltare utenti e committenti attraverso interviste o attraverso la lettura di documentazione precedentemente redatta.

Esempi di documenti da analizzare sono realizzazioni preesistenti e procedure aziendali.

Inoltre è poi importante controllare fonti, documentazioni, leggi e regolamenti.

Il reperimento dei requisiti è un'attività critica, complessa e non standardizzabile. Esistono però delle regole generali che conviene seguire.

\begin{itemize}
    \item Scegliere il corretto livello di astrazione
    \item Standardizzare la struttura delle frasi
    \item Suddividere le frasi articolare
    \item Separare le frasi sui dati da quelle sulle funzioni
    \item Costruire un glossario dei termini
    \item Individuare omonimi e sinonimi
    \item Rendere esplicito il rifermento tra termini
    \item Riorganizzare le frasi per concetti
\end{itemize}



\subsection{Progettazione concettuale}

È importante iniziare da una progettazione concettuale e quindi da uno schema concettuale poiché è importante avere a mente e chiarire in maniera solida e visiva quelli che sono i concetti che dovremo memorizzare nella base dati.

Se iniziassimo subito a mettere mano alle componenti specifiche delle basi di dati sarebbe difficile capire da dove iniziare e si rischierebbe di perdersi, infatti il modello logico ha molto focus sui dati e la rappresentazione delle relazioni e delle classi.

Il modello concettuale serve a ragionare sulla realtà di interesse indipendentemente dagli aspetti realizzativi. Questo ci permette di visualizzare quella che è la realtà in maniera schematica e chiara.

Il modello concettuale permette di rappresentare le classi di oggetti di interesse e le loro correlazioni.

Lo schema che utilizzeremo per rappresentare il modello sarà uno schema \textbf{ER}\footnote{Entità Relazione}.

\chapter{Generalizzazioni}

Una generalizzazione è il processo attraverso il quale viene associato ad una varietà di elementi il medesimo significato. Con generalizzazione viene indicato anche il significato ottenuto attraverso questo processo. Con generalizzazione viene indicato sia il processo cognitivo che la conoscenza risultante da questo processo. La generalizzazione ha la funzione di attenuare una varietà degli elementi allo scopo di semplificarne la gestione.

\begin{exmp}
Un esempio esplicativo di questa situazione può essere un esempio relativo agli alberi.

Un olmo, un oleandro, un pesco, un albicocco si possono tutti generalizzare in "\textbf{Un albero}"
\end{exmp}

Matematicamente parlando, considerando $E$ generalizzazione ogni proprietà di $E$ è anche proprietà di $E_1, ... E_n$ oggetti superiori.

Ogni istanza degli oggetti superiori $E_1, ..., E_n$ è istanza di $E$ generalizazzione.

Esistono vari tipi di generalizzazioni che differiscono a seconda della complessità e aderenza tra sovraoggetti e genralizzazione.

\begin{description}
	\item[N.B.] È importante ricordare che posso avere più tipi di generalizzazioni assieme e contemporaneamente.
\end{description}


\section{Generalizzazione totale e parziale}
\subsection{Totale}
Ogni occorrenza di $E$  è anche uno tra $E_1, ..., E_n$

\begin{exmp}
 "\textbf{una persona}"
\end{exmp}

In questo tipo di generalizzazione, se \textbf{sommiamo tutti i sovra-oggetti otteniamo l'insieme completo} della generalizzazione.

Nella rappresentazione ER, una generalizzazione di questo tipo è definita con una freccia grande dal corpo pieno(solitamente nero).

\subsection{Parziale}

In questo tipo di generalizzazione se sommiamo tutti i sovra-oggetti \textbf{non} otteniamo l'insieme completo della generalizzazione
Non tutte le $E$ sono un $E_1, ..., E_n$

\begin{exmp}
Un dottore e un avvocato possono essere generalizzati in "\textbf{una persona"}, però le persone non sono tutte avvocati o dottori.
\end{exmp}

\section{Generalizzazioni esclusive e sovrapposte}

Una generalizzazione è esclusiva quando l'intersezione dei figli è vuota.

Le generalizzazioni esclusive sono le più semplici ed in ogni caso si può semplicemente passare da una generalizzazione sovrapposta ad una generalizzazione esclusiva semplicemente dividendo la generalizzazione sovrapposta in più generalizzazioni esclusive che alla bisogna possono diventare parziali.

Una generalizzazione è sovrapposta quando l'intersezione dei figli \textbf{non} è vuota
\begin{exmp}
    Un \textit{Cliente} e un \textit{Venditore} sono entrambe \textbf{Persone} ma una \textbf{persona} può essere sia un cliente che un venditore.
\end{exmp}

\section{Conclusioni}

Concludendo, una generalizzazione può essere di questi 4 tipi ovvero combinazioni delle precedenti.

\begin{itemize}
    \item totale esclusiva
    \item totale sovrapposta
    \item parziale esclusiva
    \item parziale sovrapposta
\end{itemize}

\lstlistoflistings

\end{document}
