\chapter{Algebra relazionale}

L'algebra relazionale è un linguaggio atto a definire in maniera chiara e matematica le relazioni tra i dati.

Abbiamo quindi la differenza tra:

\begin{itemize}
    \item DDL: data definition language
    \item DML: data manipulation language
\end{itemize}

Le relazioni saranno definite come il semplice prodotto cartesiano tra insiemi.

\section{Operatore Unione}
L'operatore di unione è l'unione matematica di due insiemi, si avrà quindi come risultato tutti gli elementi degli insiemi uniti togliendo i duplicati.

Questo operatore è rappresentato dal simbolo $\cup$.

\section{Operatore intersezione}
L'operatore di intersezione sono, come in matematica, gli elementi comuni agli insiemi intersecati.

Questo operatore è rappresentato dal simbolo $\cap$.
\section{Differenza}
Considerando due insiemi, $A$ e $B$, l'operazione di differenza ritorna solo gli elementi presenti in $A$ e non in $B$.

Questo operatore è rappresentato dal simbolo $\subset$.

\section{ridefinizione}

Ho una ridefinizione quando cambio un'intestazione della tabella.

In inglese questa operazione viene definita come \textit{"Rename"}.

Questo operatore è rappresentato dal simbolo $\rho $\footnote{Rho}.

\section{selezione}

Quando prendo un sottoinsieme delle tuple faccio una selezione.

Rappresento questo operatore con $\sigma $\footnote{Sigma}

\begin{exmp}

Ecco qualche selezione d'esempio:

$\sigma \text{stipendio} > 50$;

$\sigma \text{stipendio} > 50 \wedge \text{nome}=\text{"Mario"}$;
\end{exmp}

\section{proiezione}

Sto facendo una proiezione quando prendo un sottoinsieme della tupla. Le proiezioni eliminano i duplicati.


\begin{description}
	\item[N.B.] Siccome eliminano i duplicati è bene farlo sulle super-chiavi.
\end{description}

Le proiezioni si rappresentano con il simbolo matematico $\Pi$\footnote{Pi} ovvero project.

%Da espandere di molto la sezione
\section{join}

Quando faccio una join prendo 2 relazioni e unisco le tuple su attributi uguali.

Un join è completo se faccio proiezioni sugli operandi riottengo i primi, non è completo se alcuni campi non corrispondono.

Il join ri rappresentano con il simbolo $\bowtie$.

Esistono poi più tipologie di join:
\begin{itemize}
    \item join esterno
    \item semi join
    \item join cartesiano, detto anche join naturale
    \item theta join
\end{itemize}



\subsection{Join esterno}

In un join esterno mantengo gli $R_1$ senza match mettendo i corrispondenti a \textit{NULL}

\begin{exmp}
\[ R_1 \bowtie _{LEFT} R_2  \]
\end{exmp}

\subsection{Semi join}

\begin{exmp}
\[ R_1 \bowtie _{SEMI} R_2 = \Pi _R1\left( R_1 \bowtie R_2 \right)   \]
\end{exmp}

\subsection{Join cartesiano, detto anche join naturale}

Join senza match tra attributi

\subsection{Theta join}
\[ R_1 \bowtie _{COND} R_2 = \sigma cond \left(R_1 \bowtie R_2 \right) \]
