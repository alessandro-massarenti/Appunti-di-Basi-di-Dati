\chapter{SQL}

SQL sta per Structured Query Language. Questo linguaggio serve a definire dei comandi da inviare al DBMS, esso ci risponderà con informazioni relative alla query.

Tramite l'SQL si possono inserire, rimuovere e modificare dati, inoltre tramite questo linguaggio si possono creare le strutture e aggiungere legami tra tabelle e dati.

\section{Create table}

Crea una tabella, ovvero uno schema di relazione tra domini e dati.

\begin{lstlisting}
CREATE TABLE nome (
    nome_attr TIPO VINCOLI,
    FOREIGN KEY(attributo) REFERENCES Tabella(attributo),
    UNIQUE(attributo),
    )
\end{lstlisting}

\subsection{Domini elementari}
I domini elementari sono i domini base che può avere ogni dato inserito.

\begin{itemize}
    \item Stringhe
    \begin{itemize}
        \item char
        \item varchar
    \end{itemize}
    \item number
    \begin{itemize}
        \item int
        \item float
        \item ecc
    \end{itemize}
\end{itemize}

\subsection{Domini customizzati}

Posso anche creare domini customizzati, li creo con la seguente sintassi.

\begin{lstlisting}
CREATE DOMAIN nome
AS TIPO DEFAULT valore
CHECK(condizione)

\end{lstlisting}


\subsection{Vincoli}

Posso avere vari vincoli da aggiungere ai domini e ai dati presenti nella tabella.

\begin{lstlisting}
NOT NULL
UNIQUE
PRIMARI KEY
CHECK
REFERENCES
FOREIGN KEY
\end{lstlisting}


\subsection{Update e Delete}

Posso inoltre decidere cosa succede per i dati gerarchicamente inferiori in caso di aggiornamento o rimozione dati. Per scrivere questa decisione la sintassi è la seguente:

\begin{lstlisting}
ON <update | delete> <cascade | set null | set default | no action>
\end{lstlisting}

\section{Interrogazione del DBMS}

\begin{lstlisting}
    SELECT attributi
    FROM tabela
    WHERE condizione
\end{lstlisting}

Gli attributi sono rinominabili per poter rendere più leggibile il risultato dell'interrogazione. Per rinominarli si usa:

\begin{lstlisting}
    attributo as nomeattributo
\end{lstlisting}

Questi comandi SQL trovano le loro controparti in algebra relazionale come segue:

\begin{itemize}
    \item attributi sono $\Pi$
    \item condizione sono $\sigma$
    \item rinominazione è $\rho$
\end{itemize}

\begin{description}
	\item[N.B.] Una differenza molto importante tra SQL ed algebra relazionale è che nelle select non c'è collasso, se desidero questa funzionalità in più dovrò utilizzare il la keyword DISTINCT.
 \end{description}

\subsection{Pattern di ricerca}
Quando faccio un'interrogazione del database e utilizzo il modificatore WHERE, posso utilizzare dei pattern per la ricerca, uno dei più utilizzati è \textit{LIKE}, si utilizza come segue:

\begin{lstlisting}
    nome LIKE "pattern"
\end{lstlisting}

\begin{itemize}
    \item Il simbolo "-" significa, "qualsiasi carattere"
    \item Il simbolo "\%" significa, "qualsiasi sequenza di caratteri(anche vuota)
\end{itemize}

Qualasiasi confronto con \textit{NULL} è falso.

\subsection{join}

Una clausula di join si usa per combinare le righe di due o più tabelle, basandosi su una colonna di relazione tra loro.

In algebra relazionale questo concetto è definito da $\bowtie$.


\subsubsection{join esplicito}

Posso eseguire left join, right join e full join

\subsection{order by}

In un modello relazionale, i dati non hanno ordine. Quando si esegue una query si può quindi decidere di dare ad essi un ordine per renderli più leggibili agli umani. Per riordinare i dati la keyword che si utilizza è \textit{ORDER BY}.

Tramite \textit{ORDER BY} è possibile riordinare colonna per colonna, si potrà poi decidere se in ordine crescente o decrescente.

\subsection{operatori}

Alcuni operatori utilizzabili in SQL sono: 

\begin{itemize}
    \item \textit{COUNT}
    \item \textit{MIN}
    \item \textit{MAX}
    \item \textit{AVG}
    \item \textit{SUM}
\end{itemize}

Questi operatori si può abbastanza semplicemente capire cosa fanno.

\subsubsection{Operatori aggregati e \textit{NULL}}

Gli operatori aggregati ignorano i valori \textit{NULL}, anche count ignora i valori nulli.

\subsection{Raggruppamenti}
Tramite SQL è inoiltre possibile aggregare e raggruppare i risultati delle query, per farlo si usa \textit{GROUP BY}

\subsubsection{Union}


\subsubsection{Except}
\subsubsection{Intersect}
\subsubsection{In}
\subsubsection{Any}
\subsubsection{Exist}

\section{Concetti avanzati di SQL}

\subsection{Vincoli di integrità generici}

Tramite SQL posso creare dei vincoli di integrità generici, questi vincoli controlleranno che non ci siano condizioni non rispettate.

Per utilizzare questo strumento tramite SQL si utilizza:

\begin{lstlisting}[caption= Utilizzo di CHECK]
    CHECK (<condizione>)
    
    --oppure ad esempio
    
    CHECK(lordo = netto + ritenute)
\end{lstlisting}

\subsection{Creazione di viste}

In SQL si può svuiluppare il concetto di vista che si è osservato al capitolo \ref{chap:viste}.

Per creare una vista tramite SQL la dicitura è la seguente:

\begin{lstlisting}[caption=Creazione di viste]
    CREATE VIEW <nome_vista>[<lista_attributi>] AS SELECT
    
    WITH CHECK OPTION
    
    --Il check permette update,
    --ovvero la tupla modificata rimane nella lista.
    
    SELECT * FROM <nome_vista>
\end{lstlisting}

\subsection{Coalesce}

\subsection{null if}

\subsection{CASE}

\subsection{Controllo degli accessi}
