\chapter{Normalizzazione}

La normalizzazione è un processo utilizzato per trasformare una base di dati non progettata in maniera corretta in una che sia tale.

La normalizzazione si può inoltre usare per capire se la progettazione effettuata è stata corretta.

\section{Anomailie}

Se una base di dati non è normalizzata possono sussistere varie anomalie che possono precludere la consistenza e la solidità di una base di dati.

Ridondanza
Anomalia di aggiornamento
Anomalia di cancellazione
(Come effetto collaterale rischio di prendere anche altre informazioni che avrei dovuto mantenere)

\section{Dipendenze funzionali}

Una base di dati avrà sempre intrinsecamente delle dipendenze funzionali, queste dipendenze definiscono che se A avrà un certo valore allora B avrà sempre un valore relativamente ad A

\[ A \rightarrow B\]

Impiegato -> Stipendio.

Le anomalie sono legate alle dipendenze funzionali.

Le dipendenze migliori sono quelle dove la parte di sinistra è superchiave della relazione.

\section{Forme normali}
\subsection{Forma normale di Boyce Codd(BCNFF)}

Una relazione è in forma normale di BC se ogni dipendenza funzionale non banale è buona.

Per ogni dipendenza xy che viola bcnf creo una relazione xy e tolgo y dalla tabella di partenza.

\subsection{Terza forma normale}

La terza forma normale da delle condizioni meno stringenti rispetto alla forma normale di Boice Codd.

\subsection{Svolgere gli esercizi}