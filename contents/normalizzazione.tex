\chapter{Normalizzazione}

La normalizzazione è un processo utilizzato per trasformare una base di dati non progettata in maniera corretta in una che sia tale.

La normalizzazione si può inoltre usare per capire se la progettazione effettuata è stata corretta.

\section{Anomailie}

Se una base di dati non è normalizzata possono sussistere varie anomalie che possono precludere la consistenza e la solidità di una base di dati.

\begin{itemize}
    \item Ridondanza
    \item Anomalia di aggiornamento
    \item Anomalia di cancellazione
    \item Anomailia di inserimento
\end{itemize}

\paragraph{Ridondanza}
Ho che uno stesso dato viene ripetuto più volte nella tabella.

\paragraph{Anomalia di aggiornamento}
Se un dato ridondante varia è necessario che quella ridondanza venga manuta consistente in tutte le tuple uguali. Altrimenti rischio inconsistenze che non sono facilmente gestibili.

\paragraph{Anomalia di cancellazione}
Se cancello una tupla e a quella tupla sono legate delle informazioni che non vorrei cancellare ma che essendo costruita male la struttura della tabella devo cancellare, ho perdita di dati.

\paragraph{Anomalia di inserimento}
Se ho un dato da voler inserire che non conosce gli altri campi dati della tupla non posso inserirlo perchè mancano delle informazioni che non posso avere.

In poche parole queste anomalie derivano dal fatto che in una stessa tabella sono state aggiunte troppe informazioni e non viene rispettato il principio della singola responsabilità.

\section{Dipendenze funzionali}

Una base di dati avrà sempre intrinsecamente delle dipendenze funzionali, queste dipendenze definiscono che se A avrà un certo valore allora B avrà sempre un valore relativamente ad A

\[ A \rightarrow B\]

\begin{exmp}
    $\text{Impiegato} \rightarrow \text{Stipendio}$
\end{exmp} 

Se ho una dipendenza funzionale gestita in maniera non corretta avrò una delle anomalie precedentemente descritte.

Matematicamente una dipendenza funzionale è definita come:

\[\forall t_1,t_2 \in R(X) : \pi_Y(t_1) = \pi_Y(t_2) \implies \pi_Z(t_1) = \pi_Z(t_1) \]

%Le dipendenze migliori sono quelle dove la parte di sinistra è superchiave della relazione.

Una dipendenza funzionale buona non causa anomalie.

Infatti, se la dipendenza funzionale ha a sinistra della dipendenza una chiave, allora nella parte destra avrò un dato che è responsabilità della tabella immagazzinare. Questo viene però definito in maniera più completa nelle forme normali.

\section{Forme normali}

Le forme normali definiscono un pattern per strutturare la gestione delle dipendenze funzionali in modo da evitare le anomalie.

\subsection{Forma normale di Boyce Codd(BCNFF)}

Una relazione è in forma normale di BC se ogni dipendenza funzionale non banale è buona.

Questa forma normale da inoltre uno strumento per rendere una relazione priva di anomalie.

Per ogni dipendenza xy che viola bcnf creo una relazione xy e tolgo y dalla tabella di partenza.

\subsection{Terza forma normale}

La terza forma normale da delle condizioni meno stringenti rispetto alla forma normale di Boice Codd.

\subsection{Svolgere gli esercizi}

\begin{exmp}
    Data la relazione 
    \[R(A,B,C,D)\]
    Con dipendenze funzionali
    \[ C \rightarrow D,C \rightarrow A,B \rightarrow C \]
    
    \begin{enumerate}
        \item Mostrare tutte le chiavi di R e motivare perché ognuna è chiave.
        \item Dire quali dipendenze violano la forma normale di Boyce Codd spiegandone la ragione.
        \item Decomporre in BCNF
    \end{enumerate}
    
    \paragraph{1.} Per trovare le chiavi dobbiamo per prima cosa calcolare le chiusure, sicuramente andremo a calcolare le chiusure di tutte le parti sinistre delle dipendenze. ovvero C e B

    Calcolare le chiusure significa cercare relativamente ad ogni attributo dove si può andare a finire seguendo le frecce.

    \begin{equation} \label{eq1}
        \begin{split}
            C^+&= \left\{ C \right\} \\
            &= \left\{ C,D \right\}  \\
            &= \left\{ C,D,A \right\}  \\
        \end{split}
    \end{equation}

    Quindi qui ad esempio da C potevamo andare in A e da C potevamo andare in d

    \begin{equation} \label{eq1}
        \begin{split}
            B^+&= \left\{  B \right\}\\
            &= \left\{  B,C \right\}\\
            &= \left\{  B,C,D \right\}\\
            &= \left\{  B,C,D,A \right\}\\
        \end{split}
    \end{equation}

    Nella chiusura qui sopra invece da B siamo andati in C e da C potevamo percorrere tutte le strade della ciusura di C.

    Come risultato di questi calcoli otteniamo quindi le due chiusure.

    \[C^+= \left\{ C,D,A \right\} \] \[ B^+= \left\{  B,C,D,A \right\}\]

    Ci accorgiamo che la chiusura di \textbf{B} contiene tutti gli attributi della relazione. Per questo motivo possiamo quindi dire che B è chiave della relazione.

    \paragraph{2.}

    Pe trovare quali dipendenze violano la forma normale BCNF dobbiamo controllare quali di queste dipendenze non hanno una superchiave\footnote{Le dipendenze che rispettano la BCNF sono solo le dipendenze "Buone"} al lato sinistro della rappresentazione.

    in questo caso la nostra chiave e superchiave è $B$. 

    Possiamo quindi notare come $C \rightarrow D$ e $C \rightarrow A$ violino BCNF dato che C non è superchiave.

    Siccome abbiamo delle violazioni di BCNF dobbiamo trovare come decomporre la relazione tramite BCNF.

    \paragraph{3.} Per decomporre in BCNF dobbiamo estrarre dalla relazione tutti quegli attributi che rischiano di essere interessati dalle anomalie.

    La decomposizione inizia togliendo la parte destra delle dipendenze funzionali dove l'attributo di sinistra non è chiave.

    Abbiamo quindi che la nostra relazione $R(A,B,C,D)$ con chiave trovata prima in $B$ viene separata in
    \[R(A,\underline{B},C) , R_1(\underline{C},D)\]

    Rimuoviamo poi la seconda dipendenza funzionale e ci troviamo nella situazione:

    \[R(\underline{B},C) , R_1(\underline{C},D), R_2(\underline{C},A)\]

    Possiamo quindi accorpare $R_1,R_2$ poichè hanno la stessa chiave e la situazione finale diventa:

    \[R(\underline{B},C), R_1(\underline{C},D,A)\]

\end{exmp}
