\chapter{Generalizzazioni}

Una generalizzazione è il processo attraverso il quale viene associato ad una varietà di elementi il medesimo significato. Con generalizzazione viene indicato anche il significato ottenuto attraverso questo processo. Con generalizzazione viene indicato sia il processo cognitivo che la conoscenza risultante da questo processo. La generalizzazione ha la funzione di attenuare una varietà degli elementi allo scopo di semplificarne la gestione.

\begin{exmp}
Un esempio esplicativo di questa situazione può essere un esempio relativo agli alberi.

Un olmo, un oleandro, un pesco, un albicocco si possono tutti generalizzare in "\textbf{Un albero}"
\end{exmp}

Matematicamente parlando, considerando $E$ generalizzazione ogni proprietà di $E$ è anche proprietà di $E_1, ... E_n$ oggetti superiori.

Ogni istanza degli oggetti superiori $E_1, ..., E_n$ è istanza di $E$ generalizazzione.

Esistono vari tipi di generalizzazioni che differiscono a seconda della complessità e aderenza tra sovraoggetti e genralizzazione.

\begin{description}
	\item[N.B.] È importante ricordare che posso avere più tipi di generalizzazioni assieme e contemporaneamente.
\end{description}


\section{Generalizzazione totale e parziale}
\subsection{Totale}
Ogni occorrenza di $E$  è anche uno tra $E_1, ..., E_n$

\begin{exmp}
 "\textbf{una persona}"
\end{exmp}

In questo tipo di generalizzazione, se \textbf{sommiamo tutti i sovra-oggetti otteniamo l'insieme completo} della generalizzazione.

Nella rappresentazione ER, una generalizzazione di questo tipo è definita con una freccia grande dal corpo pieno(solitamente nero).

\subsection{Parziale}

In questo tipo di generalizzazione se sommiamo tutti i sovra-oggetti \textbf{non} otteniamo l'insieme completo della generalizzazione
Non tutte le $E$ sono un $E_1, ..., E_n$

\begin{exmp}
Un dottore e un avvocato possono essere generalizzati in "\textbf{una persona"}, però le persone non sono tutte avvocati o dottori.
\end{exmp}

\section{Generalizzazioni esclusive e sovrapposte}

Una generalizzazione è esclusiva quando l'intersezione dei figli è vuota.

Le generalizzazioni esclusive sono le più semplici ed in ogni caso si può semplicemente passare da una generalizzazione sovrapposta ad una generalizzazione esclusiva semplicemente dividendo la generalizzazione sovrapposta in più generalizzazioni esclusive che alla bisogna possono diventare parziali.

Una generalizzazione è sovrapposta quando l'intersezione dei figli \textbf{non} è vuota
\begin{exmp}
    Un \textit{Cliente} e un \textit{Venditore} sono entrambe \textbf{Persone} ma una \textbf{persona} può essere sia un cliente che un venditore.
\end{exmp}

\section{Conclusioni}

Concludendo, una generalizzazione può essere di questi 4 tipi ovvero combinazioni delle precedenti.

\begin{itemize}
    \item totale esclusiva
    \item totale sovrapposta
    \item parziale esclusiva
    \item parziale sovrapposta
\end{itemize}