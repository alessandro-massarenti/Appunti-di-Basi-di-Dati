\chapter{Viste}\label{chap:viste}

Le viste sono una rappresentazione differente degli stessi dati, servono per rendere più leggibile l'informazione proveniente da alcuni dati e nei DBMS aiuta ad ottimizzare l'accesso in lettura alle informazioni, poichè in caso di join queste vengono precaricare.

\section{Tipi di viste}

Di viste ne esistono principalmente due tipi, queste viste hanno ognuna pregi e difetti.

\begin{itemize}
    \item viste materializzate
    \item viste virtuali
\end{itemize}

\subsection{Viste materializzate}
Sono viste memorizzate nel database e non hanno ricalcoli. Sono tendenzialmente delle viste ridondanti e non sempre sono supportate.

\subsection{Viste virtuali}
Sono alias per delle query e vengono ricalcolate.

\section{Ottimizzazione delle viste}

Chiaramente Esistono modi buoni e modi pessimi per utilizzare le viste, vediamo quindi come ottimizzare il loro utilizzo.

\subsection{Equivalenza di espressione}

$X(Y+Z) = X Y + X Z$

Ottimizzazione query

$ \sigma _{cond} \left( R_1 \bowtie  R_2 \right) = R_1 \bowtie \sigma _{cond} R_2$ se cond è indipendente


%Non ho capito cosa si intende in questo caso.


\subsection{Modifiche tramite viste}
È possibile fare modifiche tramite viste solo su join complete.